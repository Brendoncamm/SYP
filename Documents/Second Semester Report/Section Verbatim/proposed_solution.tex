%This file will discuss solutions implemented
%This file should be included in doc using \input{file}

\section{Proposed Solutions}

\subsection{Simulations}

\subsection{Physical Implementation}

The proposed solution incorporates the following components for the physical system architecture.

\begin{itemize}
\item Joystick Controller
\item Base Station Laptop
\item Raspberry Pi 3 Computer
\item Arduino Micro-Controller
\item 10 Degree of Freedom Inertial Measurement Unit
\item Raspberry Pi Camera Module (Future)

The system architecture is implemented as depicted in the block diagram shown in Figure \ref{fig:sys_arch}. 

\begin{figure}[H]
	\centering
	\includegraphics[width=0.7\textwidth]{flowchart_architecture.jpg}
	\caption{System Architecture}
	\label{fig:sys_arch}	
\end{figure}

A joystick controller was used as the control input for its non-spring loaded throttle control. The drone requires an altitude set point from the control input, a spring loaded control input would make the implementation of a constant set point difficult. The base station laptop hosts the GUI and the communication channel software between the USB connected joystick control input and the Raspberry Pi Wi-Fi connection. The Python scripts that enable the communication channels use the sockets library for a server, client Wi-Fi channel and pygame library to acquire control set points from the joystick controller. 


\subsection{Graphical User Interface}

