%This file will introduce the major components of the project
%This file should be included in doc using \input{file}

\section{Introduction}

\subsection{Background & Significance}

%Background of project and the significance of project.
%To be included in main file using \input{file}

Machine learning is defined as the science of getting computers to act without being explicity asked. This is achieved by the development of computer programs that can teach themselves to grow and change when exposed to different sets of data. There are two traditional types of machine learning algorithms: Batch learning algorithms and on-line machine learning algorithms. Batch learning algorithms require a set of predefined training data that is shaped over the period of time to train the model that the algorithm is running on. On-line learning uses an initial guess model that forms co-variates from that initial guess then passes them through the algorithm to form an evolved model a new set of covariates are formed from the evolved model and then fed back to make a new prediction. The loop runs continuously so that the evolved model is constantly growing and learning to adapt to certain situations. Dr. Rhinelander's research is concerned with on-line machine learning algorithms therefore the drone we are developing will configured to adapt with these algorithms. 

Quadrotor drones have been on the rise in popularity in the last several years due to their simplistic mechanical design and many practical uses. The application of these drones vary from a hobbyist flying around their neighbourhood to military personnel carrying out high risk missions. A video recording device of some sort is generally attached to the drone and the video feed is relayed to a basesation for the operator to gain a field of view (FOV) of an area of interest. Having the capability to have a continuous video feed allows the drone to be used for many practical applications including but not limited to: Traffic condition monitoring and surveillance missions. While these drones are very sophisticated and advanced devices they are are missing one aspect that is very important to further Dr. Rhinelanders research: They are not totally configurable. 

As mentioned, Dr.Rhinelanders research is concerned with on-line machine learning algorithms and without a platform that is completely configurable his research would be limited. Before the machine learning algorithms are implemented onto the drone it must first be able to be controlled. This is where we come in, we have been tasked by Dr. Rhinelander to develop a flight controller that recieves control inputs over Wi-Fi. Having a completely open source flight controller will allow for the addition of the machine learning algorithms to the flight controller software so that the drone can learn to partially, and eventually fully fly on it's own and make intelligent decisions.

\subsection{Simulations}

%Dylan



\subsection{Physical Implementation}


The physical implementation of the drone software and hardware requires a communication channel between a control input, a base station host, a wireless communication receiving unit, and a host for the controlling system. The objective of the physical implementation is to provide the software and hardware design required to successfully implement a quadrotor drone flight controller. It is desired to have a controller that is capable of sensing disturbances in flight and stabilizing the system based upon the measured disturbances.

The objectives of the physical implementation for the project begin with successfully creating a WiFi communication channel between a controller input and a wireless device placed on the drone hardware. The drone is to have flight sensing capabilities, interfaced with both the on board wireless device and a grounded controlling station. The flight sensing is to include yaw, pitch, roll and altitude measurements with a polling rate sufficient for real time control of the drone. The flight sensing is to be combined with a set point from the wirelessly transmitted control input by a PI or PID loop to control the flight of the drone with stabilizing features. The code written for the control input, communication interface, and finally the controller is to be well documented to ensure ease of use and simplicity for updating and modifying after handing over the final product.


\subsection{Graphical User Interface}


The Graphical User Interface (GUI) requires a network connection between the base station host it lives on and the Raspberry Pi 3 to recieve the serialized dictionaries containing sensor data and controller information. The objective of the GUI is to provide the user with a medium to access information regarding the current flight, such as, the altitude at which the drone has flown or how the physical controller is configured. It is desired to have a real time plot of the altitude based on the sensor information and the ability to initialize the physical controller. 

The objectives for the GUI for the project start with selecting an appropriate development framework that will function across Windows, Mac OS and Unix environments.  The GUI is to have the ability to initialize the communications between the base station host and Raspberry Pi 3 to allow for the control inputs to be sent using the base station host as well as receiving the data that is to be displayed on the GUI. The GUI is to be intuitive to avoid any unneccessary confusion with the end user. The code written and any other software used for the GUI will be provided and well documented to ensure that in event that the end user would like to modify anything after receiving the final product, this can be done effortlessly. 
  
