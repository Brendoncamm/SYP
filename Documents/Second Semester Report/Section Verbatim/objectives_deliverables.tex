%This section discusses the objectives and deliverables for the project.
%To be included in my document using \input{file}
\section{Objectives and Deliverables}
\subsection{Overall Objective}
The main objective of the project is to develop a programmable flight controller that responds appropriately to control inputs and disturbances. The flight controller will recieve control inputs over Wi-Fi from a base station. The base station can be any device that is Wi-Fi enabled and has the appropriate software installed. The software on the base station will be a graphical user interface (GUI) that allows the user to send control inputs to the drone and view statistics of the drone during operation. 

The objectives of the project have been broken down into what will be accomplished in the first semester (Short term objectives) and what will be accomplished in the second semester (Long term objectives). These objectives along with specific details of each can be viewed in sections 2.2 and 2.3 respectively.  

\subsection{Short Term Objectives (October 2016 - December 2016)}
\subsubsection{Simulation}

The simulations will allow us to gain an understanding of how the controller will respond to specific inputs. The simulation can then be tuned until the output is within the constraints set by Dr. Rhinelander. We will be simulating both the flight dynamics and controller using MATLAB and Simulink exclusively.

\subsubsection{Construction of the Drone}

The drone parts will arrive separately and assembly will be required. The extent of the assembly will be to attach the 4 Brushless Direct Current (BLDC) motors and batteries to the base of the drone. On top of the assembly the preliminary layout of the required hardware will be decided on. The layout is subject to change as we begin the final assembly in the second semester. 

\subsubsection{Initial Design of the Controller} 
A preliminary design of the flight controller will be constructed in software. 

\subsubsection{Initial Testing} 
The initial design of the controller will be tested using the brushless direct current (BLDC) motors  supplied by Dr. Rhinelander. The initial tests will allow us to gain insight on what changes to the flight controller must be main in order to meet the constraints. Along with the testing of the controller tests will be conducted to characterize the BLDC motor and electronic speed controller (ESC) 

\subsection{Long Term Objectives (Dec 2016 - April 2017)}
\subsubsection{Graphical User Interface Design}
The graphical user interface (GUI) will be installed on any base station intended to be able to operate the drone. The key features of the GUI include: A means to access the controller, displays drones position (Coordinates and altitude) and the ability the load a new build onto the drone. Some minor features will include: displays the current software build on the drone, current flight time and total flight time.
 
\subsubsection{Base Station Configuration}
This will entail installing the GUI onto the base station and configuring the base station network adapter to be able to communicate with the raspberry-pi on the drone. 

\subsubsection{Network Tests}
The intent of the network test is to gain an understanding of the network strength at various distances. A preliminary idea of how these tests will be conducted is to ping the raspberry-pi from the base station to see the time of response at these distances. 

\subsubsection{Final Controller Tests}
The final controller tests will be identical to the tests run during the first semester of the project. The purpose of these tests will be to verify that the appropriate changes were made in order to meet the defined constraints so that flight tests of the drone may be completed. 

\subsubsection{Flight Tests}
The flight tests will be run to validate that each constraint has been met to the best of our ability. The tests include, but are not limited to: looking at the drones response to disturbances, response to control inputs, response to loss of communication and verify that the GUI is reporting the expected data. 

\subsection{Deliverables}
The final deliverables will be as follows: 
\begin{itemize}
	
\item Raspberry Pi based communication interface software
\item Arduino based flight controller program
\item Base station software that relays control signal to quadcopter as well as displaying status information in a GUI
\item Detailed documentation for architecture of overall and of the programming of every subsystem.

\end{itemize}
\subsubsection{Raspberry Pi based communication interface software}
The flight controller will consist of both a Arduino microcontroller and Raspberry Pi 3 microprocessor that is Linux based. In order to establish communication between the two a script must be run. The script to achieve to communication will be supplied to Dr. Rhinelander. 
\subsubsection{Arduino based flight controller program}
The flight controller software will be well commented to ensure that it is made clear how the software works. 
\subsubsection{Detailed documentation for software}
The documentation will contain specific details regarding the software architecture as well as any specific details about how to run or load the software onto a platform. The documentation will also include instructions  on how to use the provided GUI as well as any troubleshooting techniques if any are required.  
\subsubsection{GUI}
The GUI will be loaded on to a Universal Serial Bus (USB) (USB will be supplied by Brendon Camm) so that it may be installed on any device that Dr. Rhinelander  desires to act as a base station. The GUI will be able to be installed on Linux, Windows and MAC devices.


 



	
