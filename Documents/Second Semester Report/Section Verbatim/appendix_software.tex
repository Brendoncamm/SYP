
%Preliminary Results Appendix
%Include in main doc using \input{file}
\section{Software Packages}

\subsection{Arduino Library}
\counterwithin{figure}{section}
\lstset{basicstyle=\tiny}
\begin{lstlisting}[language=C,caption={Drone.h Arduino Header File},label={lst:Drone.h}]


/*
  Drone.h - Library for Drone Controller.
  Created by Lucas Doucette, February 17, 2017.
*/
#ifndef Drone_h
#define Drone_h

#include "Arduino.h"
#include <Wire.h>
#include <Servo.h>
#include <Adafruit_Sensor.h>
#include <Adafruit_LSM303_U.h>
#include <Adafruit_BMP085_U.h>
#include <Adafruit_L3GD20_U.h>
#include <Adafruit_10DOF.h>

class Drone
{
  public:
    Drone();
    float get_sensorRoll();
    float get_sensorPitch();
    float get_sensorYaw();
    float get_sensorAltitude(float startingPressure);
    float get_currentPressure();
    float PID_Calculate(float Setpoint, float SenseRead, float kp, float kd, float ki );
    void read_PS4Setpoints();
    void initSensors();
    void initESCs(int MotorPin1, int MotorPin2,int MotorPin3,int MotorPin4);
    float read_float();
    void send_float();

};

#endif

\end{lstlisting}



\newpage


\counterwithin{figure}{section}
\begin{lstlisting}[language=C,caption={Drone.cpp Arduino Library File},label={lst:Drone.cpp}]


/*
  Drone.h - Library for Drone Controller.
  Created by Lucas Doucette, February 17, 2017.
*/

#include "Arduino.h"
#include "Drone.h"
#include <Wire.h>
#include <Servo.h>
#include <Adafruit_Sensor.h>
#include <Adafruit_LSM303_U.h>
#include <Adafruit_LSM303.h>
#include <Adafruit_BMP085_U.h>
#include <Adafruit_L3GD20_U.h>
#include <Adafruit_10DOF.h>
#include <LiquidCrystal.h>


  /* Assign a unique ID to the sensors */
Adafruit_10DOF dof   = Adafruit_10DOF();
Adafruit_LSM303_Accel_Unified accel = Adafruit_LSM303_Accel_Unified(30301);
Adafruit_LSM303_Mag_Unified   mag   = Adafruit_LSM303_Mag_Unified(30302);
Adafruit_BMP085_Unified       bmp   = Adafruit_BMP085_Unified(18001);


sensors_event_t accel_event;
sensors_event_t mag_event;
sensors_event_t bmp_event;
sensors_vec_t   orientation;

float temperature;

Drone::Drone()
{

}

float Drone::get_sensorRoll()
{
    
    float sensorRoll;


  // Calculate pitch and roll from the raw accelerometer data

  accel.getEvent(&accel_event);
  if (dof.accelGetOrientation(&accel_event, &orientation))
  {
    // 'orientation' should have valid .roll and .pitch fields
    sensorRoll=orientation.roll;
    return sensorRoll;


  }
}

float Drone::get_sensorPitch()
{

  float sensorPitch;
    
  // Calculate pitch and roll from the raw accelerometer data

  accel.getEvent(&accel_event);
  if (dof.accelGetOrientation(&accel_event, &orientation))
  {
    // 'orientation' should have valid .roll and .pitch fields

     sensorPitch=orientation.pitch;
     return sensorPitch;

  }
}

float Drone::get_sensorYaw()
{
    
  float sensorYaw;

  // Calculate the heading using the magnetometer
  mag.getEvent(&mag_event);
  if (dof.magGetOrientation(SENSOR_AXIS_Z, &mag_event, &orientation))
  {
    // 'orientation' should have valid .heading data now
    sensorYaw=orientation.heading;
    return sensorYaw;

  }
}

float Drone::get_sensorAltitude(float startingPressure)
{
    
    float sensorAltitude, NowPressure;
    
    // kalman filtering variables

    float Variance=0.108923435;
    float varProcess = 1e-12;
    float Pc = 0.0;
    float G = 0.0;
    float P = 1.0;
    float Xp = 0.0;
    float Zp = 0.0;
    float Xe = 0.0;
   
    // Calculate the altitude using the barometric pressure sensor
    bmp.getEvent(&bmp_event);
    NowPressure=bmp_event.pressure;

    // Get ambient temperature in C
    bmp.getTemperature(&temperature);


    // Convert atmospheric pressure, SLP and temp to altitude
    sensorAltitude=bmp.pressureToAltitude(startingPressure,NowPressure,temperature);


    // kalman filter;
    Pc=P+varProcess;
    G=Pc/(Pc+Variance);
    P=(1-G)*Pc;
    Xp=Xe;
    Zp=Xp;
    Xe=G*(sensorAltitude-Zp)+Xp;


    //Create Deadzone (+/- 30cm)

     // if(Xe<0.3 && Xe>-0.3)
    //{
      //  Xe=0;
    //}

    return Xe;


}

//must be called after initSensors();

float Drone::get_currentPressure()
{
    float currentPressure;
    
   //Initializing Pressure, Filtered for accuracy
   //Smooting Constant
    
   const int numReadings=100;
   float SmoothingVariable=0;
   float NowPressure=0;

    bmp.getEvent(&bmp_event);
    for(int i=0; i<numReadings; i++)
    {
        SmoothingVariable+=bmp_event.pressure;
        //delay(1);

    }

    currentPressure = SmoothingVariable/numReadings;
    SmoothingVariable=0;

    return currentPressure;

}

float Drone::PID_Calculate(float Setpoint, float SenseRead, float kp, float kd, float ki)
{
    //Define PID Variables
    float Error, DError, IError, LastTime, LastError, Output;

    //find current time
    unsigned long NowTime = millis();

    //calculate PID Error values
    Error = Setpoint-SenseRead;
    DError = (Error-LastError)/(NowTime-LastTime);
    IError += (Error*(NowTime-LastTime));

    //Calculate Output
    Output = kp*Error+ki*IError+kd*DError;

    //Save LastTime and LastError
    LastTime=NowTime;
    LastError=Error;

    
    //Requirement based on Simulink
    
    Output=(Output/9000)*(1860-1127);
    
    if(Output>=1860){
        
        Output = 1860; 
        
    }
    
    return Output;
}


void Drone::read_PS4Setpoints(float *PS4Yaw, float *PS4Pitch, float *PS4Roll, float *PS4Altitude)

{
    
    //Write serially to Pi to begin transmission.

    byte XMIT = 00000001;

    if (Serial.available())
    {
       Serial.write(XMIT);
    }


    
    while (Serial.available())
    {
        for(int i=0; i<4; i++)
        {
            *PS4Yaw=read_float();
        
        }

        for(int i=4; i<8; i++)
        {
            *PS4Pitch=read_float();

        }

        for(int i=8; i<12; i++)
        {
            *PS4Roll=read_float();
            
        }

        for(int i=12; i<16; i++)
        {
           *PS4Altitude=read_float();
            
        }

    }
    
    


}


void Drone::initSensors()
{

  if(!accel.begin())
  {
    /* There was a problem detecting the LSM303 ... check your connections */
    Serial.println(F("Ooops, no LSM303 detected ... Check your wiring!"));
    while(1);
  }
  if(!mag.begin())
  {
    /* There was a problem detecting the LSM303 ... check your connections */
    Serial.println("Ooops, no LSM303 detected ... Check your wiring!");
    while(1);
  }
  if(!bmp.begin())
  {
    /* There was a problem detecting the BMP180 ... check your connections */
    Serial.println("Ooops, no BMP180 detected ... Check your wiring!");
    while(1);
  }

}

void Drone::initESCs(Servo esc_1, Servo esc_2, Servo esc_3, Servo esc_4)
{
    

  //Initialization of the ESCs
  esc_1.writeMicroseconds(1860);
  esc_2.writeMicroseconds(1860);
  esc_3.writeMicroseconds(1860);
  esc_4.writeMicroseconds(1860);
  delay(3000);
  esc_1.writeMicroseconds(1060);
  esc_2.writeMicroseconds(1060);
  esc_3.writeMicroseconds(1060);
  esc_4.writeMicroseconds(1060);
  delay(3000);


}

void Drone::send_float (float arg) {
  byte * data = (byte *) &arg;
  Serial.write(data, sizeof(arg));
}

float Drone::read_float () {
 union{
   float a;
   unsigned char bytes[4];
 } data;
 for (int i=0; i<4; i++) {
   data.bytes[i] = Serial.read();
 }
 float test = data.a;
 return(test);
}



\end{lstlisting}

\newpage

\subsection{Raspberry Pi Scripts}
\counterwithin{figure}{section}
\lstset{basicstyle=\tiny}
\begin{lstlisting}[language=C,caption={Server.py Raspberry Pi Communication Server},label={lst:server.py}]


import socketserver
import sys
import arduino
import threading
import queue
from subprocess import check_output

#TODO:
#   Control Side
#       Write Server w/ logging
#       Test
#   Backchannel
#       Write Handler
#       Test
#   Read
#       TCP Binding

class SerialRequestHandler(threading.Thread):
    def __init__(self, stateq):
        super(self.__class__, self).__init__()
        self.stateq = stateq

    def run(self):
        #TODO: Add poison pill to exit thread
        sbus = arduino.Arduino_Controller(9600)
        while True:
            ready = sbus.ready()
            if ready:
                sbus.serial_bus.read(ready)
                sbus.serial_bus.write(self.stateq.get())


class QuadControlHandler(socketserver.BaseRequestHandler):
    def __init__(self, request, client_address, server, stateq):
        self.stateq = stateq
        super(self.__class__, self).__init__(request, client_address, server)
        return

    def handle(self):
        if self.stateq.full():
            self.stateq.get() #Queue has size of 1, if full clear for new state
        recv = self.request.recv(16)
        self.stateq.put(recv)
        print('Received: ' + str(recv))
        return


class QuadControlServer(socketserver.TCPServer):
    def __init__(self, server_address, RequestHandlerClass):
        super(self.__class__, self).__init__(server_address, RequestHandlerClass)
        self.controller = arduino.Arduino_Controller(9600)
        self.stateq = queue.Queue(1)
        return

    def serve_forever(self, poll_interval=0.5):
        thread = SerialRequestHandler(self.stateq)
        thread.start()
        super(self.__class__, self).serve_forever(poll_interval)
        return

    def finish_request(self, request, client_address):
        self.RequestHandlerClass(request, client_address, self, self.stateq)



if __name__ == '__main__':
    if sys.version_info[0] < 3:
        raise Exception('Version Error', 'Not compatible with Python version 2')

    HOST = check_output(['hostname', '-I']).strip()
    CONTROL_PORT = 2222
    COMMS_PORT = 4444

    test_serv = QuadControlServer((HOST, CONTROL_PORT), QuadControlHandler)
    test_serv.serve_forever()

\end{lstlisting}

\newpage

\begin{lstlisting}[language=C,caption={arduino.py Raspberry Pi Serial Communication Library},label={lst:arduino.py}]

# 2017-01-13 Auth: Dylan

import serial

class Arduino_Controller(object):
    #  Provides a wrapper for communicating with the Arduino over I2C/SMBUS.
    def __init__(self, baud):
        self.baudrate = baud
        self.serial_bus = serial.Serial('/dev/ttyACM0', self.baudrate)

    def write_axes(self, axes):
        self.serial_bus.write(axes)

    def ready(self):
        return self.serial_bus.inWaiting()

    def write_button(self, button):
        pass

\end{lstlisting}

\newpage

\subsection{Base Station Communication Script}
\counterwithin{figure}{section}
\lstset{basicstyle=\tiny}
\begin{lstlisting}[language=C,caption= {Control Signal and Client Script},label={lst:ps4_controller.py}]

#! /usr/bin/env python
# -*- coding: utf-8 -*-
#
# This file presents an interface for interacting with the Playstation 4 Controller
# in Python. Simply plug your PS4 controller into your computer using USB and run this
# script!
#
# NOTE: I assume in this script that the only joystick plugged in is the PS4 controller.
#       if this is not the case, you will need to change the class accordingly.
#
# Copyright � 2015 Clay L. McLeod <clay.l.mcleod@gmail.com>
#
# Distributed under terms of the MIT license.

#TODO:
#   rewrite connection for new server
#   test

# import os
# import pprint
import pygame
import socket
import struct
import sys

if sys.version_info[0] < 3:
    raise Exception('Lucas', 'not compatible with Python version 2')


class PS4Controller(object):
    """Class representing the PS4 controller. Pretty straightforward functionality."""

    controller = None
    axis_data = None
    button_data = None
    hat_data = None

    def __init__(self, axis_order=[1, 2, 3, 4], hostname='raspberrypi', port=2222):
        """Initialize the joystick components"""

        pygame.init()
        pygame.joystick.init()
        self.controller = pygame.joystick.Joystick(0)
        self.controller.init()
        self.hostname = hostname
        self.port = port
        if isinstance(axis_order, list):
            self.axis_order = axis_order  # For changing how controller axes are bound
        else:
            raise Exception(TypeError, 'axis_order must be list.')

    def update_axes(self, axis_order):
        self.axis_order = axis_order

    def listen(self):
        """Listen for events to happen"""

        if not self.axis_data:
            self.axis_data = {0: float(0),
                              1: float(0),
                              2: float(0),
                              3: float(0),
                              4: float(-1),
                              5: float(-1)}  # Added explicity number of axes to avoid waiting for input

        if not self.button_data:
            self.button_data = {}
            for i in range(self.controller.get_numbuttons()):
                self.button_data[i] = False

        if not self.hat_data:
            self.hat_data = {}
            for i in range(self.controller.get_numhats()):
                self.hat_data[i] = (0, 0)


        # host = '192.168.2.19' #ip of Server (PI)
        host = socket.gethostbyname(self.hostname)  # if fails install samba on pi and reboot

        while True:
            for event in pygame.event.get():
                if event.type == pygame.JOYAXISMOTION:
                    self.axis_data[event.axis] = round(event.value, 2)
                elif event.type == pygame.JOYBUTTONDOWN:
                    self.button_data[event.button] = True
                elif event.type == pygame.JOYBUTTONUP:
                    self.button_data[event.button] = False
                elif event.type == pygame.JOYHATMOTION:
                    self.hat_data[event.hat] = event.value

                # Insert your code on what you would like to happen for each event here!
                # In the current setup, I have the state simply printing out to the screen.

                # Defining Variables to send through the socket to the RPi, need to be strings

                # axis_data=str(self.axis_data)
                # button_data = str(self.button_data)
                # hat_data = str(self.hat_data)

                # Sending Data over a socket to the RPi
                # print(str(self.axis_data))
                # Isolate desired Axes

                axes_data = [self.axis_data[self.axis_order[0]],
                             self.axis_data[self.axis_order[1]],
                             self.axis_data[self.axis_order[2]],
                             self.axis_data[self.axis_order[3]]]
                byte_data = []  # To hold the axes data serialized to bytes
                for axis in axes_data:
                    byte_data.append(struct.pack("f", axis))  # F for float

                xmission_bytes = bytes().join(byte_data)
                connection = socket.socket()
                connection.connect((host, self.port))
                connection.send(xmission_bytes)  # sending the controller data over the port
                connection.close()
                # print(xmission_bytes)

                # os.system('cls')
                # break
                # s.send(button_data)
                # s.send(hat_data)
                # s.close()


if __name__ == "__main__":
    ps4 = PS4Controller()
    # ps4.init()
    ps4.listen()


\end{lstlisting}



\newpage




