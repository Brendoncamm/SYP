%This file will discuss the design process & additional routes that could have been taken
%This file should be included in doc using \input{file}

\section{Methods}

\subsection{Planning and Scheduling}




\subsection{Simulations}




\subsection{Physical Implementation}

The physical implementation of the system began with determining a method to interface a controller input with a base station host. To do this, research of available python libraries was performed. It was determined quickly that the most effective route would be to use Python's Pygame library which analyses a bluetooth or usb connected device and determines the types of control inputs applicable for the device. Initially, a PS4 bluetooth connected controller was successfully utilized, receiving all axis and button inputs through the Python interface. 

The next challenge of the physical implementation was to determine the best method of Wi-Fi communication. It was determined that Python's socket library would be best suited, and a Raspberry Pi 3 with a Wi-Fi module available would act as the server in the server, client architecture. A preliminary communication server client script set was written and tested successfully between the base station laptop and the Raspberry Pi server host. 

For the controller host, two possible solutions were considered. Utilizing an Arduino micro-controller as a permanent host of the control loop, communicating with the Raspberry Pi to receive control input data or using the Raspberry Pi for both communication and the controller. Using Arduino would provide a dedicated control loop and an intuitive interface to prototype with. The Raspberry Pi is capable of running the control loop, but issues with regards to a real time operating system were anticipated. Using an Arduino would require more mounting space on the drone as well as another communication channel between the RPi and Arduino to troubleshoot. Many other controllers could have been considered but both RPi and Arduino are owned by group members.








\subsection{Graphical User Interface}
\subsubsection{Framework Selection}

There is a multitude of frameworks available for graphical user interface development so in order to narrow down the choices the following restrictions were placed: 
\begin{itemize}
	\item Must function on Windows, Mac OS and UNIX environments
	\item Have the ability to display live plots
	\item Be compatibile with the physical controller initialization software
	\item Have well documented libraries for ease of development 
\end{itemize}
With these restrictions placed and further research into various frameworks the selections narrowed down to Qt and PyQt5. Each of these use the Qt framework which is regarded as a reliable, cross-platform GUI development software with extremely well documented libraries which met the major requirements. Although both Qt and PyQt5 use the same framework there are a few key differences. 

The most notable difference at first glance is that the two use different languages, Qt uses C++ where PyQt5 uses Python but the biggest difference is in the design suites. When developing using Qt you're able to use the QtCreator design suite which allows the user to design the look of the GUI using click and drag widgets, code the functionality of the widgets then simply compile all within the design suite. When using PyQt5 there isn't a complete design suite, instead the user must design the user interface using QtDesigner and then import the GUI file into the Python script, from here the widgets are coded is relatively the same other than the obvious differences between C++ and Python. 




