\section{GUI}
This section gives the user instructions on how to make use of the functionalities the GUI has to offer as well as outlines how to add additional functionality if desired. It begins by explaining the required software and libraries the user must have installed in order to use the GUI. Which definitions the pushbuttons on the GUI are linked to will be outlined and finally a brief set of instructions on how to add new pushbuttons, pages and where to find additional Qt resources will be included. 

\subsection{Required Software}
The following is a list of the required software needed to run the GUI:

\begin{itemize}
	\item QtDesigner (Included in QtCreator5.6)
	\item Python3
\end{itemize}
If your system does not have QtDesigner installed follow these steps:
\begin{enumerate}
	\item Open your preferred browser and navigate to: https://www.qt.io/qt5-6/ and click "Download"
	\item On the next page select "In-house deployment, private use, or student use" and click "Get Started"
	\item On the next page select "No" and click "Get Started"
	\item On the next page select "No" and click "Get Started"
	\item On the next page select "Desktop/multiscreen application" and click "Get Started"
	\item You should now be on a entirely different page outlining the Commerical and Open Source versions of Qt scroll down and select "Get your open source package"
	\item On the next page click "Download Now" to download the installer
	\item Once you have downloaded the installer run it
	\item With the installer now running click "Next"
	\item Enter your Qt account information, if you do not already have a Qt account you can make one within the installer enter this information and then click "Next"
	\item You should now be on the Setup page on the installer click "Next"
	\item Browse for an installation folder or click "Next" to use the default one provided, ensure the "Associate common file types with QtCreator" box is selected 
	\item On the next page click "Deselect All" at the bottom of the page and then select "Qt5.6" and "Tools" from the list. then click "Next"
	\item On the next page select "I agree" and then click "Next"
	\item On the next page click "Next"
	\item Finally, click "Install"
\end{enumerate}
If your system does not have Python3 installed follow these steps:

\begin{enumerate}
	\item Open your preferred browser and navigate to https://www.python.org/downloads/
	\item Click on "Python 3.x" where x is the version number. The version number is irrelevant as long as the first number is 3. 
	\item Scroll to the bottom of the page to "Files" and select the appropriate version for your operating system. For example, I am using a 64bit Windows OS I would click on "Windows x86-64 executable installer". Click on this to download the installer. Once it is downloaded run the installer
	\item On the installer, click "Customize installation"
	\item On the next page ensure every box is checked
	\item On the next page ensure the following boxes are checked: "Associate files with Python", "Create shortcuts for installed applications", "Add Python to environment variables" and "Precompile standard library". Once this is done click "Install"
\end{enumerate}

Once you have successfully installed QtCreator and Python3 you must install the following libraries using the "pip" command. This is done through the command window, open the command window and type, for example: "pip install PyQt5", this will install the latest version of PyQt5. Follow this process for each of the libraties listed below.

\begin{itemize}
	\item PyQt5
	\item pyqtgraph
	\item numpy
	\item sip
\end{itemize}

\subsection{Basic Functionality}
This section outlines the functionality of the Pushbutton's within the GUI

\subsubsection{Home Page}
\begin{itemize}
	\item \textbf{Start}: Initializes communications between the Base Station and Raspberry Pi
	\item \textbf{Finish}: Closes the GUI
\end{itemize}

