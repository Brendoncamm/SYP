\section{Arduino Library}
\counterwithin{figure}{section}
\lstset{basicstyle=\tiny}
\begin{lstlisting}[language=C,caption={Drone.h Arduino Header File},label={lst:Drone.h}]


/*
  Drone.h - Library for Drone Controller.
  Created by Lucas Doucette, February 17, 2017.
*/
#ifndef Drone_h
#define Drone_h

#include "Arduino.h"
#include <Wire.h>
#include <Servo.h>
#include <Adafruit_Sensor.h>
#include <Adafruit_LSM303_U.h>
#include <Adafruit_BMP085_U.h>
#include <Adafruit_L3GD20_U.h>
#include <Adafruit_10DOF.h>

class Drone
{
  public:
    Drone();
    float get_sensorRoll();
    float get_sensorPitch();
    float get_sensorYaw();
    float get_sensorAltitude(float startingPressure);
    float get_currentPressure();
    float PID_Calculate(float Setpoint, float SenseRead, float kp, float kd, float ki );
    float PID_FWD_EULER_Calculate(float Setpoint, float SenseRead, float kp, float kd, float ki, float Ts, float N);
    float Altitude_Calibration(float AltitudePID);
    float PR_Calibration(float PR_PID);
    void read_PS4Setpoints(float *PS4Yaw, float *PS4Pitch, float *PS4Roll, float *PS4Altitude);
    void initSensors();
    void initESCs(Servo esc_1, Servo esc_2, Servo esc_3, Servo esc_4);
    float read_float();
    void send_float(float arg);
    float ping_ultrasonic();
};

#endif

\end{lstlisting}



\newpage


\counterwithin{figure}{section}
\begin{lstlisting}[language=C,caption={Drone.cpp Arduino Library File},label={lst:Drone.cpp}]

/*
  Drone.h - Library for Drone Controller.
  Created by Lucas Doucette, February 17, 2017.
*/

#include "Arduino.h"
#include "Drone.h"
#include <Wire.h>
#include <Servo.h>
#include <Adafruit_Sensor.h>
#include <Adafruit_LSM303_U.h>
#include <Adafruit_LSM303.h>
#include <Adafruit_BMP085_U.h>
#include <Adafruit_L3GD20_U.h>
#include <Adafruit_10DOF.h>
#include <LiquidCrystal.h>

  /* Assign a unique ID to the sensors */
Adafruit_10DOF dof   = Adafruit_10DOF();
Adafruit_LSM303_Accel_Unified accel = Adafruit_LSM303_Accel_Unified(30301);
Adafruit_LSM303_Mag_Unified   mag   = Adafruit_LSM303_Mag_Unified(30302);
Adafruit_BMP085_Unified       bmp   = Adafruit_BMP085_Unified(18001);

sensors_event_t accel_event;
sensors_event_t mag_event;
sensors_event_t bmp_event;
sensors_vec_t   orientation;

float temperature;

Drone::Drone()
{

}

float Drone::ping_ultrasonic()
{
    

  long duration, m;

    
    
    //send low high low to ensure a clean ping
    
  pinMode(7, OUTPUT);
  digitalWrite(7, LOW);
  delayMicroseconds(2);
  digitalWrite(7, HIGH);
  delayMicroseconds(5);
  digitalWrite(7, LOW);
  
    
  pinMode(8, INPUT);
  duration = pulseIn(8, HIGH);
    
  
  //29 microseconds per cmeter, and round trip  
  m=(duration/29)/2;
    
    if(m>1500)
    {
        
        m=0;
        
    }
    
  return m;
    

    
    
}

float Drone::get_sensorRoll()
{
    
    float sensorRoll;


  // Calculate pitch and roll from the raw accelerometer data

  accel.getEvent(&accel_event);
  if (dof.accelGetOrientation(&accel_event, &orientation))
  {
    // 'orientation' should have valid .roll and .pitch fields
    sensorRoll=orientation.roll;
    return sensorRoll;


  }
}

float Drone::get_sensorPitch()
{

  float sensorPitch;
    
  // Calculate pitch and roll from the raw accelerometer data

  accel.getEvent(&accel_event);
  if (dof.accelGetOrientation(&accel_event, &orientation))
  {
    // 'orientation' should have valid .roll and .pitch fields

     sensorPitch=orientation.pitch;
     return sensorPitch;

  }
}

float Drone::get_sensorYaw()
{
    
  float sensorYaw;

  // Calculate the heading using the magnetometer
  mag.getEvent(&mag_event);
  if (dof.magGetOrientation(SENSOR_AXIS_Z, &mag_event, &orientation))
  {
    // 'orientation' should have valid .heading data now
    sensorYaw=orientation.heading;
    return sensorYaw;

  }
}



float Drone::get_sensorAltitude(float startingPressure)
{
    
    
    float sensorAltitude, NowPressure;
    
    // kalman filtering variables

    float Variance=0.108923435;
    float varProcess = 1e-12;
    float Pc = 0.0;
    float G = 0.0;
    float P = 1.0;
    float Xp = 0.0;
    float Zp = 0.0;
    float Xe = 0.0;
    
    
   
    // Calculate the altitude using the barometric pressure sensor
    bmp.getEvent(&bmp_event);
    NowPressure=bmp_event.pressure;

    // Get ambient temperature in C
    bmp.getTemperature(&temperature);


    // Convert atmospheric pressure, SLP and temp to altitude
    sensorAltitude=bmp.pressureToAltitude(startingPressure,NowPressure,temperature);


    // kalman filter;
    Pc=P+varProcess;
    G=Pc/(Pc+Variance);
    P=(1-G)*Pc;
    Xp=Xe;
    Zp=Xp;
    Xe=ping_ultrasonic()/100;

    if (Xe >= 2.0)
    {
        //Enable barometric pressure above 2 meters
        Xe=G*(sensorAltitude-Zp)+Xp;
    }

       
    return Xe;
        
    
        
}

//must be called after initSensors();

float Drone::get_currentPressure()
{
    float currentPressure;
    
   //Initializing Pressure, Filtered for accuracy
   //Smooting Constant
    
   const int numReadings=1;
   float SmoothingVariable=0;
   float NowPressure=0;

    bmp.getEvent(&bmp_event);
    for(int i=0; i<numReadings; i++)
    {
        SmoothingVariable+=bmp_event.pressure;
        //delay(1);

    }

    currentPressure = SmoothingVariable/numReadings;
    SmoothingVariable=0;

    return currentPressure;

}

float Drone::PID_Calculate(float Setpoint, float SenseRead, float kp, float kd, float ki)
{

    //Define PID Variables
    float Error, DError, IError, LastTime, LastError, Output;

    //find current time
    unsigned long NowTime = millis();

    //calculate PID Error values
    Error = Setpoint-SenseRead;
    DError = (Error-LastError)/(NowTime-LastTime);
    IError += (Error*(NowTime-LastTime));

    //Calculate Output
    Output = kp*Error+ki*IError+kd*DError;

    //Save LastTime and LastError
    LastTime=NowTime;
    LastError=Error;

    
    //Requirement based on Simulink
    
    Output=(Output/9000)*(1860-1127);
    
    if(Output>=1200){
        
        Output = 1200; 
        
    }
    
    if(Output<=1127)
    {
        Output=1127;
    }
    
    return Output;
    
    
}


float Drone::PID_FWD_EULER_Calculate(float Setpoint, float SenseRead, float kp, float kd, float ki, float Ts, float N)
{
    
    float e2, e1, e0, u2, u1, u0; // Variables for PID computation
    float a0, a1, a2, b0, b1, b2, ku1, ku2, ke0, ke1, ke2;
    
    
    // hand calculated forward euler coefficients
    a0 = (1 - N*Ts);
    a1 = (N*Ts - 2);
    a2 = (1);
    b0 = (kp*(1 - N*Ts)+ki*(N*Ts-1)+kd*N);
    b1 = (kp*(N*Ts-2)+ki*Ts-2*kd*N);
    b2 = (kp+kd*N);
    ku1 = a1/a0; ku2 = a2/a0; ke0 = b0/a0; ke1 = b1/a0; ke2 = b2/a0;
    
    /* Backward Euler for testing
    a0 = (1+N*Ts);    
    a1 = -(2 + N*Ts);
    a2 = 1;
    b0 = kp*(1+N*Ts) + ki*Ts*(1+N*Ts) + kd*N;
    b1 = -(kp*(2+N*Ts) + ki*Ts + 2*kd*N);
    b2 = kp + kd*N;
    ku1 = a1/a0; ku2 = a2/a0; ke0 = b0/a0; ke1 = b1/a0; ke2 = b2/a0;
    */

    //update variables
    e2=e1;
    e1=e0;
    u2=u1;
    u1=u0;
    
    //calculate error and output
    e0 = Setpoint - SenseRead;
    u0 = - ku1*u1 - ku2*u2 + ke0*e0 + ke1*e1 + ke2*e2;
    
    
    //return output
    return u0;
    
}

float Drone::Altitude_Calibration(float AltitudePID)
    
{
    AltitudePID=(AltitudePID/9000)*(1860-1127);
    
    if (AltitudePID > 1860)
    {
        AltitudePID = 1860; 
    }
    
    if (AltitudePID < 1127)
    {
        AltitudePID = 1127;
    }
    
    return AltitudePID;
 
}


float Drone::PR_Calibration(float PR_PID)
    
{
    if (PR_PID > 4500)
    {
        PR_PID = 4500;  
    }
    
    if (PR_PID < -4500)
    {
        PR_PID = -4500;
    }
    
    return (PR_PID/9000);
    
    
}
    
    

void Drone::read_PS4Setpoints(float *PS4Yaw, float *PS4Pitch, float *PS4Roll, float *PS4Altitude)

{
    
    //Write serially to Pi to begin transmission.

    byte XMIT = 00000001;

    if (Serial.available())
    {
       Serial.write(XMIT);
    }


    
    while (Serial.available())
    {
        for(int i=0; i<4; i++)
        {
            *PS4Yaw=read_float();
        
        }

        for(int i=4; i<8; i++)
        {
            *PS4Pitch=read_float();

        }

        for(int i=8; i<12; i++)
        {
            *PS4Roll=read_float();
            
        }

        for(int i=12; i<16; i++)
        {
           *PS4Altitude=read_float();
            
        }

    }
    
    


}

void Drone::initSensors()
{

  if(!accel.begin())
  {
    /* There was a problem detecting the LSM303 ... check your connections */
    Serial.println(F("Ooops, no LSM303 detected ... Check your wiring!"));
    while(1);
  }
  if(!mag.begin())
  {
    /* There was a problem detecting the LSM303 ... check your connections */
    Serial.println("Ooops, no LSM303 detected ... Check your wiring!");
    while(1);
  }
  if(!bmp.begin())
  {
    /* There was a problem detecting the BMP180 ... check your connections */
    Serial.println("Ooops, no BMP180 detected ... Check your wiring!");
    while(1);
  }

}

void Drone::initESCs(Servo esc_1, Servo esc_2, Servo esc_3, Servo esc_4)
{
    

  //Initialization of the ESCs
  esc_1.writeMicroseconds(1860);
  esc_2.writeMicroseconds(1860);
  esc_3.writeMicroseconds(1860);
  esc_4.writeMicroseconds(1860);
  delay(3000);
  esc_1.writeMicroseconds(1060);
  esc_2.writeMicroseconds(1060);
  esc_3.writeMicroseconds(1060);
  esc_4.writeMicroseconds(1060);
  delay(3000);


}

void Drone::send_float (float arg) {
  byte * data = (byte *) &arg;
  Serial.write(data, sizeof(arg));
}

float Drone::read_float () {
 union{
   float a;
   unsigned char bytes[4];
 } data;
 for (int i=0; i<4; i++) {
   data.bytes[i] = Serial.read();
 }
 float test = data.a;
 return(test);
}



\end{lstlisting}



\newpage

\subsection{Arduino Main File}
\counterwithin{figure}{section}
\lstset{basicstyle=\tiny}
\begin{lstlisting}[language=C,caption={Main.ino Drone Flight Program},label={lst:main.ino}]

#include <Drone.h>
#include "TimerOne.h"

Drone drone;
Servo esc_1, esc_2, esc_3, esc_4;
int MotorPin1=6, MotorPin2=9, MotorPin3=5, MotorPin4=3;
float Yaw_Setpoint, Pitch_Setpoint, Roll_Setpoint, Altitude_Setpoint;
float Current_Yaw, Current_Pitch, Current_Roll, Current_Altitude;
float initial_pressure;
float kp_alt=17257.712, ki_alt=2752.7, kd_alt=12500, N_alt=20;
float kp_PR=-24.3498729396233, ki_PR=-15.1642535450127, kd_PR=-7.30148358076729, N_PR=44.9852981416911;
float AltitudePID, PitchPID, RollPID;
float Ts = 0.0001;
float Alt_Out, Pitch_Out, Roll_Out;
float Motor1_Output, Motor2_Output, Motor3_Output, Motor4_Output;


void setup() {

  esc_1.attach(MotorPin1);  //set up motor connections & Initialize
  esc_2.attach(MotorPin2);
  esc_3.attach(MotorPin3);
  esc_4.attach(MotorPin4);
  drone.initESCs(esc_1, esc_2, esc_3, esc_4);

  //Initialize Serial Bus and IMU Sensors

  Serial.begin(9600);
  drone.initSensors();
  initial_pressure=drone.get_currentPressure();

   //initialize timer and interrupt callback
  //Timer1.initialize(100);                      
  //Timer1.attachInterrupt(PID_Interrupt);
   
   
  
}


//Commented until troubleshooting is complete
/*void PID_Interrupt(){

AltitudePID = drone.PID_FWD_EULER_Calculate(Altitude_Setpoint, Current_Altitude, kp, kd, ki, Ts);

 esc_1.writeMicroseconds(AltitudePID);
 esc_2.writeMicroseconds(AltitudePID);
 esc_3.writeMicroseconds(AltitudePID);
 esc_4.writeMicroseconds(AltitudePID);
  
}

*/
void loop() {


//Get Control Setpoints
if (Serial.available()==0)
{
  Serial.print('c');
}

while (Serial.available()>0)
  {
    Yaw_Setpoint = drone.read_float();
    Pitch_Setpoint = drone.read_float();
    Altitude_Setpoint = drone.read_float();
    Roll_Setpoint = drone.read_float(); 
  }
  
/*
Altitude_Setpoint = 0.50;
Pitch_Setpoint=0;
Roll_Setpoint=0;
*/

//Gather IMU Data
Current_Yaw = drone.get_sensorYaw();
Current_Pitch = drone.get_sensorPitch();
Current_Altitude = drone.get_sensorAltitude(initial_pressure);
Current_Roll = drone.get_sensorRoll();

//Calculate PID's and calibrate outputs based on simulink requirements.

AltitudePID = drone.PID_FWD_EULER_Calculate(Altitude_Setpoint, Current_Altitude, kp_alt, kd_alt, ki_alt, Ts, N_alt);
Alt_Out = drone.Altitude_Calibration(AltitudePID);

PitchPID = drone.PID_FWD_EULER_Calculate(Pitch_Setpoint, Current_Pitch, kp_PR, kd_PR, ki_PR, Ts, N_PR);
Pitch_Out=drone.PR_Calibration(PitchPID);

RollPID = drone.PID_FWD_EULER_Calculate(Roll_Setpoint, Current_Roll, kp_PR, kd_PR, ki_PR, Ts, N_PR);
Roll_Out = drone.PR_Calibration(RollPID);


//PID Combination as required by simulink

Motor1_Output = Alt_Out + Pitch_Out - Roll_Out;
Motor2_Output = Alt_Out - Pitch_Out - Roll_Out;
Motor3_Output = Alt_Out + Pitch_Out + Roll_Out;
Motor4_Output = Alt_Out - Pitch_Out + Roll_Out;


//Output PID Results to motors

 esc_1.writeMicroseconds(Motor1_Output);
 esc_2.writeMicroseconds(Motor2_Output);
 esc_3.writeMicroseconds(Motor3_Output);
 esc_4.writeMicroseconds(Motor4_Output);


}

\end{lstlisting}