%This file discusses our proposed approach and validation methods.
%This file should be icluded in main document with \input{file}
\section{Proposed Approach and Validation}
We have decided to approach the design of the controller by breaking it down into two phases, each phase will consist of four months. The first phase will deal purely with simulation and proof of concept testing of the hardware to gain a better understanding of what we were working with. The second phase will be to apply what we've learned during the first phase to develop the flight controller software.

\subsection{First Phase}
Before we can begin developing the flight controller we must first decide on which type of contol will be optimal for our application. The types of control we will consider are Proportional Integral and Proportional Integral Derivative. Each type of control will be applied to a simple simulation that outputs an the reaction torque of a 1kg body to a step input, which ever control supplies the most steady value of non-zero average torque will be the control type we will move forward with. Upon deciding which control type we will be using more advanced simulations will be run.

The more advanced simulation will consist of a 3D representation of the drone that is connected to various subsystems. The response of the system to a step input will be studied and the system will be tuned to obtain a reasonable response. The response we're going to achieve will simulate that the drone is hovering in a location, this response should be resemblant to that of a sinusoidal function. Upon completion of the simulations Proof of Concept (POC) testing will be performed. 

The POC testing will consist of the following items: determining methods to control the motor speed, characteristics and limitations of the hardware and software and which communication protocol is optimal. 

The main method we will be implementing to test control of the motor speed is the apply different pulse lengths to the motor. This will be achieved by using Pulse Width Modulation (PWM) applied at different duty cycles. A script will be written that applies a voltage across a potentiometer that can then be varried to apply different duty cycles. The duty cycle will be displayed on a serial terminal to easily determine the current draw vs duty cycle characteristic of the motor. 

Upon validating that the motor speed can be controlled using PWM characterization of the Electronic Speed Controllers (ESC) and BLDC motors will be performed. The ESC tests will attempt to prove that the ESC's are capable of controlling the motor speed using a variable input supplied by a potentiometer using the PWM script. To test the lift capability of a single BLDC motor a weight will be attached to a support system with the motor and blade attached to it. The weight appartatus will be placed on a zeroized scale and then power will be applied to the motor, as the propeller speed increases the reduction in weight read from the scale will be considered the lift capability. The test will be performed at various pulse width's to form a current draw versus pulse width characterization of the motor with a specified weight attached to it. 

To establish communication between the Raspberry Pi 3 and Arduino as well as determine which communication protocol (Wi-Fi or Bluetooth) will be optimal for our application basic tests will be run. 


 
