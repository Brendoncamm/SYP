%Preliminary Results Appendix
%Include in main doc using \input{file}
\section{Proof of Concept and Testing}

\begin{figure}[h]
  \centering
  \includegraphics[width=0.5\textwidth]{Motor_Characterization.jpg}
  \caption{Motor Characterization Testing}
  \label{fig:Motor_Char}
\end{figure}

\begin{figure}[h]
  \centering
  \includegraphics[width=0.5\textwidth]{Motor_Driving_Schematic.jpg}
  \caption{Motor Driver Circuit POC Schematic}
  \label{fig:Motor_Drive_Schem}
\end{figure}

\begin{lstlisting}

#include<Servo.h>

//Define Constants

Servo esc;
const int MotorPin = 9;
const int Var_Resistor = A0;
int Resistor_Value = 0;
int time_value = 0;

void setup() {
	
	//begin serial monitor & baud rate
	Serial.begin(9600)
	
	//set up motor connection
	esc.attach(MotorPin);  
	
	//Enable Variable Resistor Input
	pinMode(Var_Resistor,INPUT); 
	
}

void loop() {
	
	//Read the resistor value 
	Resistor_Value=analogRead(Var_Resistor);
	
	//Var_Resistor will be between 1-1024
	// need to convert to 1060-1860
	//1060-1860 is a range of 800, 800/1023=0.78201
	
	//Conversion into req'd pulse width range
	time_value=(Resistor_Value*0.78201)+1060;
	
	//write the time signal to the servo motor
	esc.writeMicroseconds(time_value);
	Serial.println(time_value);
	
	
}

\end{lstlisting}
