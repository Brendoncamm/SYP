%This section discusses our preliminary results.
%This file should be included in doc using \input{file}
\section{Preliminary Results}
\subsection{Simulations Performed}
\begin{itemize}
  \item{PWM Simulation}
  \item{Thrust Simulation}
  \item{Torque Simulation}
  \item{Visual Representation}
 \end{itemize}
  
\subsubsection{PWM Simulation}

\subsubsection{Thrust Simulation}

\subsubsection{Torque Simulation}

\subsubsection{Visual Representation}

  
\subsection{Proof of Concept Testing Performed}
\begin{itemize}
  \item{PWM}
  \item{Electronic Speed Controller}
  \item{Wi-Fi Range}
  \item{I2C Communication Channel}
  \item{Playstation Controller Integration}
 \end{itemize}
  
  \subsubsection{PWM}
  
Using an arduino microcontroller, a potentiometer and a simple DC motor, a circuit was devised to test a PWM output of the arduino to drive a motor circuit. The motor circuit was isolated using a mosfet as the switching operator and a diode to ensure there would be no damaging back emf in the system. A 9V battery powered the isolated circuit.
  
Through programming, the voltage across the potentiometer was taken into the arduino as an analog input and mapped to a digital output as a PWM duty cycle. The script was successful in providing an input controlled PWM value to the motor. The code was modified to suit the needs of the electronic speed controllers by applying an input controlled pulse width as a function of time in replacement of a duty cycle as the ESC's require a pulse width range of 1060 $\mu$s and 1860 $\mu$s.
  
  \subsubsection{Electronic Speed Controller}
  
The Afro ESC 12Amp BEC UltraLite Multirotor ESC V3 was tested at St. Mary's University with the assistance of Dr. Rhinelander. An arduino running a script to map a potentiometer to a PWM duty cycle was used as an attempt to control the motor through the ESC. A 12V power supply fed the ESC while the arduino controlled the duty cycle of the speed controller producing a voltage output to control the motor speed. 

The ESC testing was successful, the testing proved the arduino's capability to control the motor with a variable input. The testing had flaws as a pwm duty cycle was used instead of a timed pulse width input. The duty cycle had potential of operating correcly as the range of times could have been calibrated to a range in the duty cycle although this method proved to be difficult due to low values causing the ESC to enter calibration mode. The script used to operate the ESC was re-written as a timed pulse width to ensure complete compatibility and ease of future integration.

  \subsubsection{Wi-Fi Range}
  
  A simple proof of concept regarding the range of Wi-Fi communications was performed. The test incorperated a a Wi-Fi communicating camera tethered to a Wi-Fi output from a cell phone. A user walked down Spring Garden holding the cell phone and found the approximate distance at which the phone and the camera lost communication. It was found that the range was approximately 100 ft with line of sight available but no Wi-Fi boosting technology.
  
  \subsubsection{I2C Communication Channel}
  
  The flight controller will consist of both an arduino microcontroller and a 
  
  
  \subsubsection{Playstation Controller Integration}
